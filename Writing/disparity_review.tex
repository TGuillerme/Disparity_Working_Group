RH: disparate disparity analyses.

\textbf{The disparate ways to analyse morphological disparity}

Thomas
Guillerme\includegraphics[width=0.4902in,height=0.1807in]{./ObjectReplacements/Object 2},
Natalie
Cooper\includegraphics[width=0.3984in,height=0.1807in]{./ObjectReplacements/Object 4},
Stephen
Brusatte\includegraphics[width=0.1937in,height=0.1807in]{./ObjectReplacements/Object 6},
Katie
Davis\includegraphics[width=0.1992in,height=0.1807in]{./ObjectReplacements/Object 8},
Andrew L
Jackson\includegraphics[width=0.1937in,height=0.1807in]{./ObjectReplacements/Object 10},
Sylvain
Gerber\includegraphics[width=0.1953in,height=0.1807in]{./ObjectReplacements/Object 12},
Anjali
Goswami\includegraphics[width=0.1929in,height=0.1807in]{./ObjectReplacements/Object 14},
Kevin
Healy\includegraphics[width=0.1953in,height=0.1807in]{./ObjectReplacements/Object 16},
Melanie
Hopkins\includegraphics[width=0.1953in,height=0.1807in]{./ObjectReplacements/Object 18},
Graeme
Lloyd\includegraphics[width=0.1953in,height=0.1807in]{./ObjectReplacements/Object 20},
Joseph
O'Reilly\includegraphics[width=0.2437in,height=0.1807in]{./ObjectReplacements/Object 22},
Abi
Pate\includegraphics[width=0.2437in,height=0.1807in]{./ObjectReplacements/Object 24},
Emilie
Rayfield\includegraphics[width=0.2437in,height=0.1807in]{./ObjectReplacements/Object 26},
Erin
Saupe\includegraphics[width=0.239in,height=0.1807in]{./ObjectReplacements/Object 28},
Emma
Sherratt\includegraphics[width=0.2425in,height=0.1807in]{./ObjectReplacements/Object 30},
Graham
Slater\includegraphics[width=0.2437in,height=0.1807in]{./ObjectReplacements/Object 32},
Gavin H
Thomas\includegraphics[width=0.2465in,height=0.1807in]{./ObjectReplacements/Object 34}
and Philip
Donoghue\includegraphics[width=0.2437in,height=0.1807in]{./ObjectReplacements/Object 36}\\
\includegraphics[width=0.1929in,height=0.1807in]{./ObjectReplacements/Object 38}\emph{School
of Biological Sciences, University of Queensland, St. Lucia, Queensland,
Australia.}\\
\includegraphics[width=0.1929in,height=0.1807in]{./ObjectReplacements/Object 40}\emph{Department
of Life Sciences, Natural History Museum, London, Cromwell Road, London,
SW7 5BD, UK.}\\
\includegraphics[width=0.2437in,height=0.1807in]{./ObjectReplacements/Object 42}\emph{Bristol}

+ These authors contributed equally to the manuscript.

*\textbf{Corresponding author: }\emph{guillert@tcd.ie}\\

\textbf{NOTE TO ALL THE CO-AUTHORS:}

\textbf{-Please write in Google docs using }\emph{\textbf{Suggesting
mode}}\textbf{ only }(upper right corner of your screen)

\textbf{-The word limit for this is 5000 words so we can't add any large
sections.}

\textbf{-There is no reference limit per se but please don't add too
many. I will sort out the references later so don't correct the
formatting of them here.}

-I suggest you add your references using the \textbf{Paperpile} addon
(go in Add-ons menu \textgreater{} Get Add-ons\ldots{} \textgreater{}
Search for Paperpile \textgreater{} Add/Install; then go in Add-ons
menus \textgreater{} Paperpile \textgreater{} Manage citation). If you
don't use Paperpile, please don't add references as just (``Someone et
al. 2000'' but add also a link to the correct Google Scholar entry in
comments).

\textbf{-Add your affiliation in the author list above}. The
affiliations count in the word count so please 1) keep it short (max 1
per person!) and 2) try to use one that's already in there. Also add any
initials I've missed in your names.

- If in doubt, check formatting here:
\href{https://royalsociety.org/journals/authors/author-guidelines/}{\emph{https://royalsociety.org/journals/authors/author-guidelines/}}

Thanks!

\textbf{Abstract (200 words max)}

Morphological disparity analysis is used to characterize and investigate
discrete patterns of variation in the shape, form, function, and the
ecology of organisms. Since their conception in the 1980s, these methods
have been used to ask a variety of questions about how and why
morphological diversity evolves, and the ecological consequences of
these changes. There are many methodological approaches for studying
disparity, however, all of these come with caveats and pitfalls that
can, if ignored, lead to misinterpretations of disparity patterns. Here
we review morphological disparity methods and their caveats, and provide
best practice guidelines for their application and interpretation. We
discuss the data required for disparity studies, as well as
methodological challenges, including multidimensionality, ordination,
metric choice, hypothesis testing, and ancestral state estimations.
There is no ``one-size-fits-all'' approach to characterizing and
investigating the evolution of morphological disparity, and the
available tools should always be used in the context of a specific
biological question that will determine data and method selection at
every stage of the analysis.

\textbf{Keywords}: multidimensionality, palaeobiology, ecology\\

1. Introduction

Variation in the shape, form and function of organisms often exhibits a
pattern of discrete clusters rather than continuous gradients. These
patterns can be pictured as multidimensional spaces encompassing all the
possible forms an organism can take, and where each individual region of
this space is the result of a unique combination of many variables. One
commonly used approach to analyse these patterns and to understand the
processes that lead to them, is to analyse the morphological diversity
of a group, often referred to as a ``morphological disparity analysis''
\href{https://paperpile.com/c/sTGYvp/0y4V}{(Deline et al. 2018)}.

The initial informal definition of morphological disparity was
``multidimensional morphological dissimilarity at a macroevolutionary
scale'' (Runnegar 1987,
\href{https://paperpile.com/c/sTGYvp/Uns3}{(Gould 2000)}). In the
seminal disparity papers of the late 1980s, disparity was used to
investigate body plan evolution after the Cambrian explosion
\href{https://paperpile.com/c/sTGYvp/CiPy}{(Briggs, Fortey, and Wills
1992)}\href{https://paperpile.com/c/sTGYvp/CidX+6tNm+oenu+TtGs}{(Gould
1991; Valentine 1986; M. Foote and Gould 1992; Morris 1989)};
\href{https://paperpile.com/c/sTGYvp/CiPy}{(Briggs, Fortey, and Wills
1992)};. Early disparity analyses focused on broad questions such as:
How does diversity evolve? Are some phenotypes more common than others?
and Can anatomical shapes evolve in all ``directions'' of the
multidimensional space
\href{https://paperpile.com/c/sTGYvp/yqPw+fTJ3}{(Mike Foote 1997,
1995)}? Since the 1990s, a great deal of literature from a range of
sub-fields in biology and palaeontology, and an equally large number of
methods, have used ``disparity'' as a broader umbrella term for studies
of multivariate variation. Although most influential reviews have
treated disparity as a pattern
\href{https://paperpile.com/c/sTGYvp/yqPw+nFf7+vTHS}{(Mike Foote 1997;
Wills 2001; Melanie J. Hopkins and Gerber 2017)}, there is currently no
consensus on whether to restrict use of the term disparity to its
original macroevolutionary-scale definition, or whether it can refer to
any study of multivariate variation. Here we use a broader definition
and suggest that morphological disparity is a \emph{pattern of variation
}in phenotype. Disparity analysis has been used to tackle a vast array
of questions. We identify four main classes of disparity analysis as
follows.

(1) \emph{Descriptive disparity} \emph{analyses. }The seminal studies of
disparity described the shapes of organisms and how they differed among
groups \href{https://paperpile.com/c/sTGYvp/fTJ3+eZ3F}{(Mike Foote 1995;
Wills, Briggs, and Fortey 1994)}. These consist of descriptive
multidimensionality analyses of the patterns in the diversity of
morphological traits, addressing questions such as: why are some
morphological traits combinations more common than others and what are
the properties of the resulting morphospace?
\href{https://paperpile.com/c/sTGYvp/fTJ3+I0Ic+QVvv}{(Mike Foote 1995;
Raup 1961; Gerber 2017)}.

(2) \emph{Disparity-through-time} \emph{analyses. }This approach
investigates how the shapes of organisms have changed through time,
focussing on the disparity of taxa in particular time bins or slices.
This approach has been widely used in palaeobiology to answer a range of
macroevolutionary questions
\href{https://paperpile.com/c/sTGYvp/PbSx+khc9+xxh5}{(Close et al. 2015;
Cisneros and Ruta 2010; Hughes, Gerber, and Wills 2013)}, such as: how
does disparity accumulate over the history of a clade
\href{https://paperpile.com/c/sTGYvp/ekU4+s33b}{(Guillerme and Cooper
2018; Wright 2017)}, and how does disparity change across mass
extinction events \href{https://paperpile.com/c/sTGYvp/EETc}{(Friedman
2010)}?

(3) \emph{Disparity and diversity. }Morphological disparity provides
another perspective on biodiversity; high morphological disparity
represents a high diversity of morphologies (i.e. shapes or body plans)
and is, presumably, associated with high levels of ecological and
functional diversity (see below). This makes disparity a more useful
measure of diversity than merely species richness. Indeed, most studies
that have investigated disparity and taxonomic diversity support an
effective decoupling
\href{https://paperpile.com/c/sTGYvp/2tbJ+geAO+hea5+aVVj}{(Fortey,
Briggs, and Wills 1996; Ruta et al. 2013; M. J. Hopkins 2013; Moyne and
Neige 2007)} - though this result breaks down at extremes of diversity;
extremely low diversity tends to yield low disparity and \emph{vice
versa}. This approach has been used to investigate whether some groups
are more successful than other in their exploration of new evolutionary
strategies @@@.

(4) \emph{Disparity and ecology. }The disparity of a group can be used
as a proxy for either the functional role it plays within an ecosystem
or its ecological niche. These analyses assume that groups with high
disparity are also likely to be functionally and ecologically diverse,
and that groups found in similar regions of shape space will have
similar functional and ecological roles
\href{https://paperpile.com/c/sTGYvp/EETc+tSIy+qjj9}{(Friedman 2010;
Pierce, Angielczyk, and Rayfield 2008; P. S. L. Anderson et al. 2011)}.
Note however, that the links between form and function are not always
clear cut; traits can be linked to multiple functions and multiple
functions can be linked to a single trait
\href{https://paperpile.com/c/sTGYvp/Ejzr}{\emph{(Wainwright et al.
2005)}}. This approach is used to investigate competitive replacements
\href{https://paperpile.com/c/sTGYvp/EeC8}{(Stephen L. Brusatte et al.
2008)} and changes in ecosystem function during and after mass
extinctions \href{https://paperpile.com/c/sTGYvp/EETc}{(Friedman 2010)}.
It is particularly common in paleobiology where it is not possible to
observe directly the ecological or functional characteristics of extinct
species. \href{https://paperpile.com/c/sTGYvp/Ejzr}{(Wainwright et al.
2005)}

Of course, these four broad categories are not independent and many
studies use a combination of approaches
\href{https://paperpile.com/c/sTGYvp/EeC8}{(Stephen L. Brusatte et al.
2008)}. Furthermore, disparity analysis is continuously expanding into
areas. While this is exciting for the field, there are several issues
that may reduce the utility of disparity analyses if they are not
tackled. Here we review these issues and present best practice
guidelines to allow researchers to optimise their analyses. We first
discuss the appropriate data required for disparity analyses, then
review various challenging methodological aspects of disparity analyses.
Throughout it is important to remember that these tools should always be
used in the context of a specific scientific question, as this will
drive data and methodological choices at every stage of the process.

\includegraphics[width=6.5in,height=4.9445in]{Pictures/10000201000002EE0000023BE5B78D4C4B79F709.png}

\emph{Figure disparities {[}figure\_disparities.pdf{]}: The four main
types of disparity analysis. Descriptive disparity analyses focus on
describing the properties of the morphospace ; Disparity-through-time
analyses investigate the evolution of the morphospace through time
including the effect of extinction events; Disparity and diversity
analysis use disparity as a proxy for species diversity; Disparity and
ecology analyses use disparity as a proxy for the ecological or
functional role of a group. These categories are not independent and
many analyses will cover more than one.}

2. Data in Disparity Analyses

Regardless of the precise approach being used, disparity analyses are
based on morphological traits. It is important, therefore, to consider
which traits to use in the context of the specific question under
investigation. Traits for disparity analyses come from three main types
of data: (1) discrete ``cladistic'' characters, i.e coding the
absence-or-presence of features or a discrete characteristic of a trait
\href{https://paperpile.com/c/sTGYvp/PbSx}{(Close et al. 2015)}); (2)
continuous measurements of features
\href{https://paperpile.com/c/sTGYvp/qjj9}{(P. S. L. Anderson et al.
2011)}; or (3) more sophisticated mathematical measurements from
geometric morphometric data (e.g. Procrustes superimpositions
\href{https://paperpile.com/c/sTGYvp/RjqE}{(Cooney et al. 2017)}, or
Fourier contours \href{https://paperpile.com/c/sTGYvp/ZEDR}{(Spriggs et
al. 2018)}). None of these are fundamentally better, however, some
questions and methodological pipelines are more appropriate with
specific types of data.

When investigating variation within bat wing shapes, for example, both
homologous landmarks from geometric morphometric analysis and continuous
measurements of bones may be appropriate to capture patterns of wing
variation. However, if the question focuses on the convergence between
bat and bird wings, the homology of the traits to measure is more
complex to define and the outline shape of the wings might more readily
answer the question from a mechanical perspective. Similarly, if the
focus is on wing function, i.e. whether the aerodynamic properties of
wings vary within bats or between bats and birds, the traits collected
should reflect these aerodynamic properties (e.g. wingspan, aspect
ratio, etc.). Whereas if the focus is on convergence between different
bats and birds, it would be preferable to use traits that have
facilitated flight in both groups (e.g. digit length, integumentary
system, etc.). Where there is any doubt about the appropriate traits to
choose, we advise using different kinds of data for the same feature to
determine whether it captures the same disparity.

This dicussion assumes that researchers are collecting their own data
for disparity analyses, but often this is not the case. It is common,
especially in disparity-through-time analyses, to recycle discrete
character data collected for use in phylogenetic inference (e.g. REFs
that have done this). This has the advantage of being quick, cheap and
efficient, but tends to artificially increase disparity between
phylogenetically distinct groups because phylogenetic characters are
collected specifically for the purpose of discriminating groups
\href{https://paperpile.com/c/sTGYvp/fTJ3}{(Mike Foote 1995)}. This
needs to be considered when interpreting results, especially as when new
clades appear, the synapomorphies of that clade will naturally lead to
an apparent shift or increase in disparity. Similarly, data used for
phylogenetic reconstruction is likely to be highly skewed towards easily
available and taxonomically relevant traits, for example teeth in
mammals. If your disparity question is related to feeding ecology or
similar then this may be appropriate, but if the focus is on dispersal
ability then it would be sensible to select only postcranial traits. It
is often tempting to use all of a large dataset to increase the power of
your analyses, but if the data are not suitable for the question at hand
then the results may be meaningless (although ssome studies have suggest
this may not be a major issue;
\href{https://paperpile.com/c/sTGYvp/ZqML+Qrba}{(Zou and Zhang 2016;
Dávalos et al. 2014)}
\href{https://paperpile.com/c/sTGYvp/xLdm}{(Melanie J. Hopkins 2017)}).

Finally, trait data suffers from the same shortcomings as most data --
data can be missing, non-overlapping, hierarchical, inapplicable,
ambiguous, polymorphic, correlated, or have insufficient sample size
\href{https://paperpile.com/c/sTGYvp/yO2t}{(Palci and Lee 2018)},
\href{https://paperpile.com/c/sTGYvp/Yrbg}{(Brazeau, Guillerme, and
Smith 2017)} {[}String of morpho data problems cite{]}. Biological
phenomena such as allometry and sexual dimorphism may also influence
trait data. More practically, data collection is constrained by the time
and money available, making collating a ``perfect'' dataset difficult.
We suggest that trait data are collected with the question in mind and
considering all of the caveats above, but where issues are unavoidable
at the very least these caveats must be considered during the analyses
and subsequent interpretation of the results.

\emph{Figure data {[}figure\_data.pdf{]}: major routes to obtain
disparity data. Data can be collected as discrete trait observations
(e.g. presence or absence data) or as continuous data. Continuous data
can be collected by different methods: linear measurements, landmark
coordinates or contours (curves). These measurement can then in turn be
mathematically transformed (logarithm, scaling, Procrustes
Superimposition, Elliptic Fourier, etc.). Regardless of the method, this
data collection results in a trait matrix where the observed traits
constitute columns and the studied elements (e.g. taxa) the rows.}

\includegraphics[width=6.5in,height=6.5in]{Pictures/10000201000007D0000007D0F8A72A6590C667D3.png}

3. Disparity Methods

Once suitable trait data have been collected, we need to consider the
disparity analysis. Below we address several key aspects including (i)
the difficulty of dealing with multidimensional data; (ii) caveats
associated with using ordination techniques to reduce the dimensionality
of the data; (iii) various metrics used to summarise the disparity of a
group; (iv) caveats related to methods that exist for hypothesis testing
within a disparity analysis framework; and (v) the specific case of
ancestral state estimations in disparity-through-time analyses. As noted
above, these tools should always be used in the context of a specific
question, as this will determine which methods are most appropriate.

\protect\hypertarget{anchor}{}{}3.1. Multidimensional mayhem

The multidimensional aspect of disparity analyses adds a layer of
technical complexity because the human brain naturally struggles to
comprehend more than three spatial dimensions. This problem is often
apparent in how people interpret the results of disparity analyses. For
example, it is standard for disparity results to use ordination
techniques to reduce their dimensionality, and for only the first or
second axes of ordination (e.g. principal components (PCs) one and two)
to be interpreted and used to draw conclusions. Changes in disparity
along these axes may show a pattern or gradient, but often do not
reflect the true changes across all the dimensions of the data
\href{https://paperpile.com/c/sTGYvp/1SD2+sN5d+xaUx+o4w7}{(Bookstein
1997, 2015, {[}b{]} 2017, {[}a{]}
20}\href{https://paperpile.com/c/sTGYvp/1SD2+sN5d+xaUx+o4w7}{1}\href{https://paperpile.com/c/sTGYvp/1SD2+sN5d+xaUx+o4w7}{7)}{[}CITE
Vera and Thomas paper{]}. Visual interpretations of multidimensional
data can also easily be misleading. For example, an obvious
discontinuity (or overlap) between two groups in a morphospace of PC1
and PC2, may not be apparent when looking at the same groups in a
morphospace of PC2 and PC3 etc. or across the whole multidimensional
space. Furthermore, multidimensional spaces are not necessary Euclidean.
Such non-Euclidean spaces often have non-intuitive properties, for
example straight lines in some dimensions are not straight in
non-Euclidean space and, even less intuitively, the distance between
points A and B might not be equal to the distance between points B and A
\href{https://paperpile.com/c/sTGYvp/SJbC}{(Gerber 2014)}.

3.2. To ordinate or not to ordinate; that is the (multidimensional)
question?

Reducing the dimensionality of a disparity dataset can be tricky.
Ordination is advantageous for plotting and visualising the data, and
can reveal properties of the morphospace not captured by disparity
metrics (see Metrics section {[}metrics{]}). Additionally, after
ordinating the data it is possible to further reduce the number of
dimensions by only selecting a certain number of axes, for example by
selecting only the axes that describe up to 95\% of the variation. In
the case of geometric morphometric data, ordination is particularly
useful as it conserves the mathematical properties of the data while
efficiently reducing the dimensions
\href{https://paperpile.com/c/sTGYvp/oFiP}{(Legendre and Legendre
2012)}. This has clear advantages for interpreting the results. For
example, the axes will represent gradients of biological variation (e.g.
elongation and flattening of the beak
\href{https://paperpile.com/c/sTGYvp/RjqE}{(Cooney et al. 2017)}).

However, various caveats can arise from such transformations of the
data. For example, in the case of an ordination in a geometric
morphometric context, not using all the axes from the ordination can
lead to misinterpretation of the results. Furthermore, interpreting
biological variation along the axes is always a \emph{post-hoc}
procedure and may have little relation to the disparity question, for
example, if the first few ordination axes represent the elongation of
the beak in birds, but the question is about wing disparity.

In many cases, ordination might not be necessary. For example, if the
disparity metric being used relies on all the data (see Metrics section
{[}metrics{]}) it is not necessary to calculate it on ordinated data
{[}e.g.\href{https://paperpile.com/c/sTGYvp/PbSx}{(Close et al.
2015)}{]}. Additionally, in some cases, reducing the dimensionality of a
dataset can render its interpretation more problematic. For example,
when the analysed data is non-Euclidean (e.g. discrete morphological
characters), ordination can be difficult, and the resulting
non-Euclidean ordinated space can be hard to interpret
\href{https://paperpile.com/c/sTGYvp/SJbC}{(Gerber 2014)}. This is
problematic when comparing the position of groups in the
multidimensional space, as the measured distances might not be
comparable. Finally, the \emph{post-hoc} interpretation of the gradient
of variation on the ordination axes can be impossible or biologically
meaningless. Although some gradients are easy to detect or interpret
(e.g. the elongation of the beak along the first axis; cooney paper),
some are not. For example, with discrete morphological data, a gradient
between the species that have many characters in state 1 and the ones
that have more in state 0 has no biological meaning. Because of these
caveats, we strongly recommend that data for multidimensional analyses
is not automatically ordinated, and that careful consideration is given
to whether the question can be answered without ordination.

\includegraphics[width=6.5in,height=5.7638in]{Pictures/10000201000002730000022C18503E8C16A21886.png}

\emph{Figure methods: Different ways to define and visualise a
morphospace: A.0: The morphospace is the collected trait space (with
some potential modification - see data section above): this can be
visualised using bi-plots (or correlation plots - B.3) between all the
traits but becomes hard to interpret after a few traits. Another
approach is to directly ordinate the trait matrix (with a PCA - A.1 -
e.g. in }\href{https://paperpile.com/c/sTGYvp/ZQS9}{\emph{(Zelditch,
Swiderski, and David Sheets 2012)}}\emph{- or a PCO - A.2; e.g. in
Brusatte et al. 2008). This results in a k (\textless{}m) space that can
also be hard to visualise even though the first axis hold most of the
variation (B.1). Another approach is to create a distance matrix (A.3;
e.g. in Close et al., 2015) this matrix can then be ordinated as well
for further analysis (A.2 and used as the ordinated morphospaces
described above) or ``flattened'' to a manageable number of dimensions
(2 or 3 - B.2).}

\protect\hypertarget{anchor-1}{}{}3.3. Disparity metrics

Because disparity is multidimensional, a large component of any
disparity analysis involves considering how to extract a meaningful (and
interpretable) summary of disparity, i.e. choosing which disparity
metric (or index)\href{https://paperpile.com/c/sTGYvp/vTHS}{(Melanie J.
Hopkins and Gerber 2017)} to use. As with any summary metric, and
because of this reduction in dimensionality, disparity metrics will only
reflect some aspects of the morphospace, never its whole complexity. It
is therefore often beneficial to use more than one metric to summarise
different aspects of interest. The choice of disparity metric is
essential to the disparity analysis and should always be driven by the
question at hand.

When considering only one dimension, disparity metrics can be used to
reflect the ``size'' of the distribution (e.g. the range, quantiles or
variance), or the most likely value (i.e. the central tendency, mean,
median or mode). Among these metrics, some will have more attractive
properties than others, such as sensitivity to outliers -- range, mean
and mode are highly sensitive whereas quantiles, variance and median are
less so -- and will thus be more or less appropriate for different
questions. For example, if the question refers to ``size'' of a group in
the morphospace (e.g. does group A occupy as much space as group B?),
metrics related to the spread of the group in the morphospace are the
most appropriate (e.g. volume
\href{https://paperpile.com/c/sTGYvp/47fI}{(Díaz et al. 2016)}, distance
from the centroid
\href{https://paperpile.com/c/sTGYvp/vTHS+yyNa}{(Melanie J. Hopkins and
Gerber 2017; Finlay and Cooper 2015)}, variance and range
\href{https://paperpile.com/c/sTGYvp/tGyd}{(S. L. Brusatte et al.
2008)}). Conversely, if the question refers to the ``position'' of a
group in the morphospace (e.g. does group A occupy the same space as
group B?), metrics related to the distance between the elements within a
group and a fixed point in the morphospace are most appropriate (e.g.
????). Finally, if the question relates to the density of elements in
some regions of the morphospace (e.g. is group A more dense than group
B?), metrics related to the pairwise distances between elements will be
most appropriate (e.g. nearest neighbour distance, pairwise distances,
etc. \href{https://paperpile.com/c/sTGYvp/PbSx}{(Close et al. 2015)}).

In addition to considering what properties of disparity the metrics
should capture, it is important to also consider the mathematical
properties of the metrics and their associated caveats
\href{https://paperpile.com/c/sTGYvp/nFf7+ROH8}{(Wills 2001; Ciampaglio,
Kemp, and McShea 2001)}. For example, measuring the full sum of the
variance of each dimension of the space does not require one to add the
covariance between the axes in a ordinated space using a principal
components analysis (PCA). This is not true however, in other
mathematical spaces or when not all dimensions or elements are
considered, even in a PCA. More simply, when looking at a group within
morphospace, the sum of variances is not mathematically equal to the sum
of the variance of each dimension in the space. Further,
multidimensional space has some counter-intuitive properties that need
to be considered such as the ``curse of dimensionality''
\href{https://paperpile.com/c/sTGYvp/Qsl3}{(Bellman 1966)}.
Product-based metrics used as proxies of volumes (e.g. product of
ranges, hypervolume, hypercube, etc.) will tend towards zero fairly
quickly for spaces with even a modest number of dimensions
{[}\href{https://paperpile.com/c/sTGYvp/Qsl3}{(Bellman 1966)}; Donoho
2000{]}. Some other types of metrics are also extremely sensitive to
outliers and can be easily biased by sample size, for example range
\href{https://paperpile.com/c/sTGYvp/aSSL}{(Butler et al. 2012)} or
convex hull based metrics
\href{https://paperpile.com/c/sTGYvp/aSSL+PwyQ}{(Butler et al. 2012;
Jackson et al. 2011)}.

\protect\hypertarget{anchor-2}{}{}3.4. Testing disparity hypotheses

As mentioned above, disparity analysis should always be used in the
context of a specific question. Once disparity metrics have been
calculated, the associated question must be tested in an appropriate
statistical framework. The multidimensional statistical toolkit for
ecology and evolution has been greatly expanded in recent years
\href{https://paperpile.com/c/sTGYvp/ZnDd}{(Adams and Collyer 2018)} but
is not routinely applied to disparity analyses. Instead, disparity
hypothesis testing is mostly confined to a small set of well-established
methods. One commonly used test is the non-parametric permutation
analysis of variance \href{https://paperpile.com/c/sTGYvp/3hy2+SC6L}{(M.
J. Anderson and Walsh 2013; M. J. Anderson 2001)}. This test is an
analysis of variance (ANOVA) of the pairwise distances between different
groups. Although statistically valid, this test is sometimes not
directly related to the hypothesis under test. For example, PERMANOVA
tests whether two groups share the same variance/covariance in a
``distance-space''. This is not the same as testing whether the two
groups overlap in morphospace. Explicitly stating the hypothesis being
tested can help with understanding which statistical test to apply for a
specific question.

It is also important to consider which data to apply a statistical test
on. For example, in morphological disparity analysis, especially for
palaeobiological questions, data are often bootstrapped. This has two
advantages: first, when the disparity metric is unidimensional, e.g. the
sum of variances, bootstrapping the data generates a distribution of the
metric that can then be analysed using the vast statistical toolkit
available for comparing distributions; second, when data are scarce,
bootstrapping the data allows users to introduce variance, rendering the
test less sensitive to outliers. However, bootstrapped data are
pseudoreplicates and thus non-independent. This violates the assumptions
of most parametric statistical tests. Furthermore, the number of
bootstrap pseudoreplicates will inevitably increase the Type I error
rate. These factors are often ignored in disparity analyses.

3.5. Phylogenetic autocorrelation

As with all comparative datasets, the data used in disparity analyses
are not independent because close relatives will tend to have more
similar morphologies than more distant relatives
\href{https://paperpile.com/c/sTGYvp/WXik}{(Harvey and Pagel 1998)}.
Thus for many disparity analyses phylogenetic relationships should be
taken into account. Multivariate phylogenetic comparative methods have
been reviewed recently by
\href{https://paperpile.com/c/sTGYvp/ZnDd}{(Adams and Collyer 2018)} and
therefore will not covered further in this review. However, we do
briefly draw attention to the fact that phylogenetic corrections of PCA
are inappropriate (cite Pennell Uyeda paper on phylo PCA).

\protect\hypertarget{anchor-3}{}{}3.6. Ancestral state estimation in
disparity-through-time analyses

One final methodological detail to consider applies only to
disparity-through-time analyses. Often these analyses use ancestral
state estimation (often mistakenly referred to as ancestral state
reconstruction) to extract disparity estimates for non-sampled taxa
and/or nodes of a phylogeny. Ancestral state estimation can be performed
at two points in the disparity analysis pipeline, either (1)
pre-ordination, i.e. the estimation is done before transformation of the
data (e.g. ordination, or distance matrix construction) and is simply
based on the original data; or (2) post-ordination, i.e. the estimation
is done after transformation of the data by estimating the ancestral
states using the transformed matrix (e.g. the ordination scores)
\href{https://paperpile.com/c/sTGYvp/53SJ}{(Lloyd 2018)}.

Pre-ordination ancestral state estimation can change the ordinated
space's geometry -- i.e. the relationship between the points are not yet
estimated -- and implies longer computational times, but once the
morphospace is defined its properties will not change. Post-ordination
ancestral state estimation will not change the ordinated space's
geometry and is faster to compute, however, it will change the
morphospace properties such as the number of parameters, i.e. elements
in the space, or the variance on each axis, which can be problematic for
statistical tests down the line
\href{https://paperpile.com/c/sTGYvp/53SJ}{(Lloyd 2018)}.

More importantly, however, using any ancestral state estimation method
has several caveats. First and foremost, ancestral state estimations are
highly dependent on the data and chosen methods. Second, when using
post-ordination ancestral state estimates, the morphospace can be
saturated leading pairwise-distance metrics to tend to zero. In general,
using ancestral state estimation can help with recovering patterns of
changes in disparity but should not be used simply to generate extra
data points to increase statistical power. In fact, these extra points
are not independent and can also generate problematic side effects,
especially when testing for the effects of mass extinctions on
disparity.

4. Disparity analyses for the future

Morphological disparity analyses are a major tool in evolutionary
biology, but they have many caveats. There, there is no
``one-size-fits-all'' morphological disparity analysis pipeline. As with
any multidimensional analysis, there are many nuances in deciding which
data to use and how to analyse it, stemming from the explicit hypothesis
being tested. We believe that many of the problems in morphological
disparity analysis can arise when ``blindly'' applying methodological
pipelines while losing sight of the biological question being tested. We
therefore advise keeping the methodology simple and tractable, at least
in the preliminary analysis. Also, many of the caveats and technical
problems in disparity analyses have been tackled in other fields, mainly
ecology. This is unsurprising, as ecological data are also often
multidimensional
\href{https://paperpile.com/c/sTGYvp/krNU+60H0}{(Donohue et al. 2013;
Canter et al. 2018)}. Seeking solutions using these literature is a
sensible next step for disparity analysis.

While morphological disparity would greatly benefit from advances in
multidimensional analysis in different fields similarly, the concept of
a morphospace could benefit other fields. For example, the
multidimensional analysis in Diaz
\href{https://paperpile.com/c/sTGYvp/47fI}{(Díaz et al. 2016)} looking
at the patterns of form and function in plants can be thought as a
morphospace or a life-history-space (an eco-space?); isotopic analyses
in \href{https://paperpile.com/c/sTGYvp/PwyQ}{(Jackson et al. 2011)} can
be represented as a isotope-space; ecosystem functioning in (Donohue et
al. 2013) as an ecosystem-space, etc. These generalisations could also
be exported for any set of traits (e.g. acousto-spaces for acoustic
traits?) and even beyond evolutionary biology (e.g. glotto-spaces for
linguistic traits?).

Although disparity analyses are now simple to implement
\href{https://paperpile.com/c/sTGYvp/xDqf+J2G1+9JdS+9Zoi+bCsU+EmTR+2KmX}{(Guillerme
2018; Adams and Otárola-Castillo 2013; Bouxin 2005; Harmon et al. 2008;
Lloyd 2016; Navarro 2003; Dixon 2003)}, it is crucial to remember that
they are multidimensional analyses; and multidimensional analyses are
complex. In Jurassic Park, Dr Ian Malcolm summarises this problem in an
elegant way: ``{[}We{]} scientists were so preoccupied with whether or
not {[}we{]} could that {[}we{]} didn't stop to think if {[}we{]}
should.'' We believe that morphological analysis in general will greatly
benefit from emphasising the methodological decisions made
(\emph{whether we should}), rather than simply using disparity analysis
because \emph{we can}.

5. Author contributions

TG, NC and PD proposed this review. All authors edited the manuscript.

6. Acknowledgments

The Royal Society. TG funding.

Donoho, D. L. (2000). High-dimensional data analysis: The curses and
blessings of dimensionality. \emph{AMS math challenges lecture},
\emph{1}(2000), 32.

References

\href{http://paperpile.com/b/sTGYvp/ZnDd}{Adams, Dean C., and Michael L.
Collyer. 2018. ``Multivariate Phylogenetic Comparative Methods:
Evaluations, Comparisons, and Recommendations.''
}\href{http://paperpile.com/b/sTGYvp/ZnDd}{\emph{Systematic
Biology}}\href{http://paperpile.com/b/sTGYvp/ZnDd}{ 67 (1): 14--31.}

\href{http://paperpile.com/b/sTGYvp/J2G1}{Adams, Dean C., and Erik
Otárola-Castillo. 2013. ``Geomorph: Anrpackage for the Collection and
Analysis of Geometric Morphometric Shape Data.''
}\href{http://paperpile.com/b/sTGYvp/J2G1}{\emph{Methods in Ecology and
Evolution / British Ecological
Society}}\href{http://paperpile.com/b/sTGYvp/J2G1}{ 4 (4): 393--99.}

\href{http://paperpile.com/b/sTGYvp/SC6L}{Anderson, Marti J. 2001. ``A
New Method for Non-Parametric Multivariate Analysis of Variance.''
}\href{http://paperpile.com/b/sTGYvp/SC6L}{\emph{Austral
Ecology}}\href{http://paperpile.com/b/sTGYvp/SC6L}{ 26 (1): 32--46.}

\href{http://paperpile.com/b/sTGYvp/3hy2}{Anderson, Marti J., and Daniel
C. I. Walsh. 2013. ``PERMANOVA, ANOSIM, and the Mantel Test in the Face
of Heterogeneous Dispersions: What Null Hypothesis Are You Testing?''
}\href{http://paperpile.com/b/sTGYvp/3hy2}{\emph{Ecological
Monographs}}\href{http://paperpile.com/b/sTGYvp/3hy2}{ 83 (4): 557--74.}

\href{http://paperpile.com/b/sTGYvp/qjj9}{Anderson, Philip S. L., Matt
Friedman, Martin D. Brazeau, and Emily J. Rayfield. 2011. ``Initial
Radiation of Jaws Demonstrated Stability despite Faunal and
Environmental Change.''
}\href{http://paperpile.com/b/sTGYvp/qjj9}{\emph{Nature}}\href{http://paperpile.com/b/sTGYvp/qjj9}{
476 (7359): 206--9.}

\href{http://paperpile.com/b/sTGYvp/Qsl3}{Bellman, R. 1966. ``Dynamic
Programming.''
}\href{http://paperpile.com/b/sTGYvp/Qsl3}{\emph{Science}}\href{http://paperpile.com/b/sTGYvp/Qsl3}{
153 (3731): 34--37.}

\href{http://paperpile.com/b/sTGYvp/1SD2}{Bookstein, Fred L. 1997.
}\href{http://paperpile.com/b/sTGYvp/1SD2}{\emph{Morphometric Tools for
Landmark Data: Geometry and
Biology}}\href{http://paperpile.com/b/sTGYvp/1SD2}{. Cambridge
University Press.}

\href{http://paperpile.com/b/sTGYvp/sN5d}{---------. 2015. ``The
Relation between Geometric Morphometrics and Functional Morphology, as
Explored by Procrustes Interpretation of Individual Shape Measures
Pertinent to Function.''
}\href{http://paperpile.com/b/sTGYvp/sN5d}{\emph{Anatomical Record
}}\href{http://paperpile.com/b/sTGYvp/sN5d}{ 298 (1): 314--27.}

\href{http://paperpile.com/b/sTGYvp/o4w7}{---------. 2017a. ``A Newly
Noticed Formula Enforces Fundamental Limits on Geometric Morphometric
Analyses.''
}\href{http://paperpile.com/b/sTGYvp/o4w7}{\emph{Evolutionary
Biology}}\href{http://paperpile.com/b/sTGYvp/o4w7}{ 44 (4): 522--41.}

\href{http://paperpile.com/b/sTGYvp/xaUx}{---------. 2017b. ``A Method
of Factor Analysis for Shape Coordinates.''
}\href{http://paperpile.com/b/sTGYvp/xaUx}{\emph{American Journal of
Physical Anthropology}}\href{http://paperpile.com/b/sTGYvp/xaUx}{ 164
(2): 221--45.}

\href{http://paperpile.com/b/sTGYvp/9JdS}{Bouxin, Guy. 2005. ``Ginkgo, a
Multivariate Analysis Package.''
}\href{http://paperpile.com/b/sTGYvp/9JdS}{\emph{Journal of Vegetation
Science: Official Organ of the International Association for Vegetation
Science}}\href{http://paperpile.com/b/sTGYvp/9JdS}{ 16 (3): 355--59.}

\href{http://paperpile.com/b/sTGYvp/Yrbg}{Brazeau, Martin D., Thomas
Guillerme, and Martin R. Smith. 2017. ``Morphological Phylogenetic
Analysis with Inapplicable Data.''
https://doi.org/}\href{http://dx.doi.org/10.1101/209775}{10.1101/209775}\href{http://paperpile.com/b/sTGYvp/Yrbg}{.}

\href{http://paperpile.com/b/sTGYvp/CiPy}{Briggs, D. E., R. A. Fortey,
and M. A. Wills. 1992. ``Morphological Disparity in the Cambrian.''
}\href{http://paperpile.com/b/sTGYvp/CiPy}{\emph{Science}}\href{http://paperpile.com/b/sTGYvp/CiPy}{
256 (5064): 1670--73.}

\href{http://paperpile.com/b/sTGYvp/tGyd}{Brusatte, S. L., M. J. Benton,
M. Ruta, and G. T. Lloyd. 2008. ``The First 50 Myr of Dinosaur
Evolution: Macroevolutionary Pattern and Morphological Disparity.''
}\href{http://paperpile.com/b/sTGYvp/tGyd}{\emph{Biology
Letters}}\href{http://paperpile.com/b/sTGYvp/tGyd}{ 4 (6): 733--36.}

\href{http://paperpile.com/b/sTGYvp/EeC8}{Brusatte, Stephen L., Michael
J. Benton, Marcello Ruta, and Graeme T. Lloyd. 2008. ``Superiority,
Competition, and Opportunism in the Evolutionary Radiation of
Dinosaurs.''
}\href{http://paperpile.com/b/sTGYvp/EeC8}{\emph{Science}}\href{http://paperpile.com/b/sTGYvp/EeC8}{
321 (5895): 1485--88.}

\href{http://paperpile.com/b/sTGYvp/aSSL}{Butler, Richard J., Stephen L.
Brusatte, Brian Andres, and Roger B. J. Benson. 2012. ``How Do
Geological Sampling Biases Affect Studies of Morphological Evolution in
Deep Time? A Case Study of Pterosaur (Reptilia: Archosauria)
Disparity.'' }\href{http://paperpile.com/b/sTGYvp/aSSL}{\emph{Evolution;
International Journal of Organic
Evolution}}\href{http://paperpile.com/b/sTGYvp/aSSL}{ 66 (1): 147--62.}

\href{http://paperpile.com/b/sTGYvp/60H0}{Canter, Erin J., Catalina
Cuellar-Gempeler, Abigail I. Pastore, Thomas E. Miller, and Olivia U.
Mason. 2018. ``Predator Identity More than Predator Richness Structures
Aquatic Microbial Assemblages in Sarracenia Purpurea Leaves.''
}\href{http://paperpile.com/b/sTGYvp/60H0}{\emph{Ecology}}\href{http://paperpile.com/b/sTGYvp/60H0}{
99 (3): 652--60.}

\href{http://paperpile.com/b/sTGYvp/ROH8}{Ciampaglio, Charles N.,
Matthieu Kemp, and Daniel W. McShea. 2001. ``Detecting Changes in
Morphospace Occupation Patterns in the Fossil Record: Characterization
and Analysis of Measures of Disparity.''
}\href{http://paperpile.com/b/sTGYvp/ROH8}{\emph{Paleobiology}}\href{http://paperpile.com/b/sTGYvp/ROH8}{
27 (4): 695--715.}

\href{http://paperpile.com/b/sTGYvp/khc9}{Cisneros, Juan Carlos, and
Marcello Ruta. 2010. ``Morphological Diversity and Biogeography of
Procolophonids (Amniota: Parareptilia).''
}\href{http://paperpile.com/b/sTGYvp/khc9}{\emph{Journal of Systematic
Palaeontology}}\href{http://paperpile.com/b/sTGYvp/khc9}{ 8 (4):
607--25.}

\href{http://paperpile.com/b/sTGYvp/PbSx}{Close, Roger A., Matt
Friedman, Graeme T. Lloyd, and Roger B. J. Benson. 2015. ``Evidence for
a Mid-Jurassic Adaptive Radiation in Mammals.''
}\href{http://paperpile.com/b/sTGYvp/PbSx}{\emph{Current Biology:
CB}}\href{http://paperpile.com/b/sTGYvp/PbSx}{ 25 (16): 2137--42.}

\href{http://paperpile.com/b/sTGYvp/RjqE}{Cooney, Christopher R., Jen A.
Bright, Elliot J. R. Capp, Angela M. Chira, Emma C. Hughes, Christopher
J. A. Moody, Lara O. Nouri, Zoë K. Varley, and Gavin H. Thomas. 2017.
``Mega-Evolutionary Dynamics of the Adaptive Radiation of Birds.''
}\href{http://paperpile.com/b/sTGYvp/RjqE}{\emph{Nature}}\href{http://paperpile.com/b/sTGYvp/RjqE}{
542 (7641): 344--47.}

\href{http://paperpile.com/b/sTGYvp/Qrba}{Dávalos, Liliana M., Paúl M.
Velazco, Omar M. Warsi, Peter D. Smits, and Nancy B. Simmons. 2014.
``Integrating Incomplete Fossils by Isolating Conflicting Signal in
Saturated and Non-Independent Morphological Characters.''
}\href{http://paperpile.com/b/sTGYvp/Qrba}{\emph{Systematic
Biology}}\href{http://paperpile.com/b/sTGYvp/Qrba}{ 63 (4): 582--600.}

\href{http://paperpile.com/b/sTGYvp/0y4V}{Deline, Bradley, Jennifer M.
Greenwood, James W. Clark, Mark N. Puttick, Kevin J. Peterson, and
Philip C. J. Donoghue. 2018. ``Evolution of Metazoan Morphological
Disparity.''
}\href{http://paperpile.com/b/sTGYvp/0y4V}{\emph{Proceedings of the
National Academy of Sciences of the United States of
America}}\href{http://paperpile.com/b/sTGYvp/0y4V}{ 115 (38):
E8909--18.}

\href{http://paperpile.com/b/sTGYvp/47fI}{Díaz, Sandra, Jens Kattge,
Johannes H. C. Cornelissen, Ian J. Wright, Sandra Lavorel, Stéphane
Dray, Björn Reu, et al. 2016. ``The Global Spectrum of Plant Form and
Function.''
}\href{http://paperpile.com/b/sTGYvp/47fI}{\emph{Nature}}\href{http://paperpile.com/b/sTGYvp/47fI}{
529 (7585): 167--71.}

\href{http://paperpile.com/b/sTGYvp/2KmX}{Dixon, Philip. 2003. ``VEGAN,
a Package of R Functions for Community Ecology.''
}\href{http://paperpile.com/b/sTGYvp/2KmX}{\emph{Journal of Vegetation
Science: Official Organ of the International Association for Vegetation
Science}}\href{http://paperpile.com/b/sTGYvp/2KmX}{ 14 (6): 927.}

\href{http://paperpile.com/b/sTGYvp/krNU}{Donohue, Ian, Owen L. Petchey,
José M. Montoya, Andrew L. Jackson, Luke McNally, Mafalda Viana, Kevin
Healy, Miguel Lurgi, Nessa E. O'Connor, and Mark C. Emmerson. 2013. ``On
the Dimensionality of Ecological Stability.''
}\href{http://paperpile.com/b/sTGYvp/krNU}{\emph{Ecology
Letters}}\href{http://paperpile.com/b/sTGYvp/krNU}{ 16 (4): 421--29.}

\href{http://paperpile.com/b/sTGYvp/yyNa}{Finlay, Sive, and Natalie
Cooper. 2015. ``Morphological Diversity in Tenrecs (Afrosoricida,
Tenrecidae): Comparing Tenrec Skull Diversity to Their Closest
Relatives.''
}\href{http://paperpile.com/b/sTGYvp/yyNa}{\emph{PeerJ}}\href{http://paperpile.com/b/sTGYvp/yyNa}{
3 (April): e927.}

\href{http://paperpile.com/b/sTGYvp/oenu}{Foote, M., and S. J. Gould.
1992. ``Cambrian and Recent Morphological Disparity.''
}\href{http://paperpile.com/b/sTGYvp/oenu}{\emph{Science}}\href{http://paperpile.com/b/sTGYvp/oenu}{
258 (5089): 1816.}

\href{http://paperpile.com/b/sTGYvp/fTJ3}{Foote, Mike. 1995.
``Morphological Diversification of Paleozoic Crinoids.''
}\href{http://paperpile.com/b/sTGYvp/fTJ3}{\emph{Paleobiology}}\href{http://paperpile.com/b/sTGYvp/fTJ3}{
21 (03): 273--99.}

\href{http://paperpile.com/b/sTGYvp/yqPw}{---------. 1997. ``THE
EVOLUTION OF MORPHOLOGICAL DIVERSITY.''
}\href{http://paperpile.com/b/sTGYvp/yqPw}{\emph{Annual Review of
Ecology and Systematics}}\href{http://paperpile.com/b/sTGYvp/yqPw}{ 28
(1): 129--52.}

\href{http://paperpile.com/b/sTGYvp/2tbJ}{Fortey, R. A., D. E. G.
Briggs, and M. A. Wills. 1996. ``The Cambrian Evolutionary `explosion':
Decoupling Cladogenesis from Morphological Disparity.''
}\href{http://paperpile.com/b/sTGYvp/2tbJ}{\emph{Biological Journal of
the Linnean Society. Linnean Society of
London}}\href{http://paperpile.com/b/sTGYvp/2tbJ}{ 57 (1): 13--33.}

\href{http://paperpile.com/b/sTGYvp/EETc}{Friedman, Matt. 2010.
``Explosive Morphological Diversification of Spiny-Finned Teleost Fishes
in the Aftermath of the End-Cretaceous Extinction.''
}\href{http://paperpile.com/b/sTGYvp/EETc}{\emph{Proceedings. Biological
Sciences / The Royal Society}}\href{http://paperpile.com/b/sTGYvp/EETc}{
277 (1688): 1675--83.}

\href{http://paperpile.com/b/sTGYvp/SJbC}{Gerber, Sylvain. 2014. ``Not
All Roads Can Be Taken: Development Induces Anisotropic Accessibility in
Morphospace.''
}\href{http://paperpile.com/b/sTGYvp/SJbC}{\emph{Evolution \&
Development}}\href{http://paperpile.com/b/sTGYvp/SJbC}{ 16 (6):
373--81.}

\href{http://paperpile.com/b/sTGYvp/QVvv}{---------. 2017. ``The
Geometry of Morphospaces: Lessons from the Classic Raup Shell Coiling
Model.'' }\href{http://paperpile.com/b/sTGYvp/QVvv}{\emph{Biological
Reviews of the Cambridge Philosophical
Society}}\href{http://paperpile.com/b/sTGYvp/QVvv}{ 92 (2): 1142--55.}

\href{http://paperpile.com/b/sTGYvp/CidX}{Gould, Stephen Jay. 1991.
``The Disparity of the Burgess Shale Arthropod Fauna and the Limits of
Cladistic Analysis: Why We Must Strive to Quantify Morphospace.''
}\href{http://paperpile.com/b/sTGYvp/CidX}{\emph{Paleobiology}}\href{http://paperpile.com/b/sTGYvp/CidX}{
17 (04): 411--23.}

\href{http://paperpile.com/b/sTGYvp/Uns3}{---------. 2000.
}\href{http://paperpile.com/b/sTGYvp/Uns3}{\emph{Wonderful Life: The
Burgess Shale and the Nature of
History}}\href{http://paperpile.com/b/sTGYvp/Uns3}{. Random House.}

\href{http://paperpile.com/b/sTGYvp/xDqf}{Guillerme, Thomas. 2018.
``dispRity : A Modular R Package for Measuring Disparity.''
}\href{http://paperpile.com/b/sTGYvp/xDqf}{\emph{Methods in Ecology and
Evolution / British Ecological
Society}}\href{http://paperpile.com/b/sTGYvp/xDqf}{ 9 (7): 1755--63.}

\href{http://paperpile.com/b/sTGYvp/ekU4}{Guillerme, Thomas, and Natalie
Cooper. 2018. ``Time for a Rethink: Time Sub-Sampling Methods in
Disparity-through-Time Analyses.''
}\href{http://paperpile.com/b/sTGYvp/ekU4}{\emph{Palaeontology}}\href{http://paperpile.com/b/sTGYvp/ekU4}{
61 (4): 481--93.}

\href{http://paperpile.com/b/sTGYvp/9Zoi}{Harmon, Luke J., Jason T.
Weir, Chad D. Brock, Richard E. Glor, and Wendell Challenger. 2008.
``GEIGER: Investigating Evolutionary Radiations.''
}\href{http://paperpile.com/b/sTGYvp/9Zoi}{\emph{Bioinformatics
}}\href{http://paperpile.com/b/sTGYvp/9Zoi}{ 24 (1): 129--31.}

\href{http://paperpile.com/b/sTGYvp/WXik}{Harvey, Paul H., and Mark D.
Pagel. 1998. }\href{http://paperpile.com/b/sTGYvp/WXik}{\emph{The
Comparative Method in Evolutionary
Biology}}\href{http://paperpile.com/b/sTGYvp/WXik}{. Oxford University
Press, USA.}

\href{http://paperpile.com/b/sTGYvp/xLdm}{Hopkins, Melanie J. 2017.
``How Well Does a Part Represent the Whole? A Comparison of Cranidial
Shape Evolution with Exoskeletal Character Evolution in the Trilobite
Family Pterocephaliidae.''
}\href{http://paperpile.com/b/sTGYvp/xLdm}{\emph{Palaeontology}}\href{http://paperpile.com/b/sTGYvp/xLdm}{
60 (3): 309--18.}

\href{http://paperpile.com/b/sTGYvp/vTHS}{Hopkins, Melanie J., and
Sylvain Gerber. 2017. ``Morphological Disparity.'' In
}\href{http://paperpile.com/b/sTGYvp/vTHS}{\emph{Evolutionary
Developmental Biology}}\href{http://paperpile.com/b/sTGYvp/vTHS}{,
1--12.}

\href{http://paperpile.com/b/sTGYvp/hea5}{Hopkins, M. J. 2013.
``Decoupling of Taxonomic Diversity and Morphological Disparity during
Decline of the Cambrian Trilobite Family Pterocephaliidae.''
}\href{http://paperpile.com/b/sTGYvp/hea5}{\emph{Journal of Evolutionary
Biology}}\href{http://paperpile.com/b/sTGYvp/hea5}{ 26 (8): 1665--76.}

\href{http://paperpile.com/b/sTGYvp/xxh5}{Hughes, Martin, Sylvain
Gerber, and Matthew Albion Wills. 2013. ``Clades Reach Highest
Morphological Disparity Early in Their Evolution.''
}\href{http://paperpile.com/b/sTGYvp/xxh5}{\emph{Proceedings of the
National Academy of Sciences of the United States of
America}}\href{http://paperpile.com/b/sTGYvp/xxh5}{ 110 (34):
13875--79.}

\href{http://paperpile.com/b/sTGYvp/PwyQ}{Jackson, Andrew L., Richard
Inger, Andrew C. Parnell, and Stuart Bearhop. 2011. ``Comparing Isotopic
Niche Widths among and within Communities: SIBER - Stable Isotope
Bayesian Ellipses in R.''
}\href{http://paperpile.com/b/sTGYvp/PwyQ}{\emph{The Journal of Animal
Ecology}}\href{http://paperpile.com/b/sTGYvp/PwyQ}{ 80 (3): 595--602.}

\href{http://paperpile.com/b/sTGYvp/oFiP}{Legendre, P., and Loic F. J.
Legendre. 2012.
}\href{http://paperpile.com/b/sTGYvp/oFiP}{\emph{Numerical
Ecology}}\href{http://paperpile.com/b/sTGYvp/oFiP}{. Elsevier.}

\href{http://paperpile.com/b/sTGYvp/bCsU}{Lloyd, Graeme T. 2016.
``Estimating Morphological Diversity and Tempo with Discrete
Character-Taxon Matrices: Implementation, Challenges, Progress, and
Future Directions.''
}\href{http://paperpile.com/b/sTGYvp/bCsU}{\emph{Biological Journal of
the Linnean Society. Linnean Society of
London}}\href{http://paperpile.com/b/sTGYvp/bCsU}{ 118 (1): 131--51.}

\href{http://paperpile.com/b/sTGYvp/53SJ}{---------. 2018. ``Journeys
through Discrete-Character Morphospace: Synthesizing Phylogeny, Tempo,
and Disparity.''
}\href{http://paperpile.com/b/sTGYvp/53SJ}{\emph{Palaeontology}}\href{http://paperpile.com/b/sTGYvp/53SJ}{
61 (5): 637--45.}

\href{http://paperpile.com/b/sTGYvp/TtGs}{Morris, S. C. 1989. ``Burgess
Shale Faunas and the Cambrian Explosion.''
}\href{http://paperpile.com/b/sTGYvp/TtGs}{\emph{Science}}\href{http://paperpile.com/b/sTGYvp/TtGs}{
246 (4928): 339--46.}

\href{http://paperpile.com/b/sTGYvp/aVVj}{Moyne, Sébastien, and Pascal
Neige. 2007. ``The Space-Time Relationship of Taxonomic Diversity and
Morphological Disparity in the Middle Jurassic Ammonite Radiation.''
}\href{http://paperpile.com/b/sTGYvp/aVVj}{\emph{Palaeogeography,
Palaeoclimatology,
Palaeoecology}}\href{http://paperpile.com/b/sTGYvp/aVVj}{ 248 (1-2):
82--95.}

\href{http://paperpile.com/b/sTGYvp/EmTR}{Navarro, Nicolas. 2003. ``MDA:
A MATLAB-Based Program for Morphospace-Disparity Analysis.''
}\href{http://paperpile.com/b/sTGYvp/EmTR}{\emph{Computers \&
Geosciences}}\href{http://paperpile.com/b/sTGYvp/EmTR}{ 29 (5):
655--64.}

\href{http://paperpile.com/b/sTGYvp/yO2t}{Palci, Alessandro, and Michael
S. Y. Lee. 2018. ``Geometric Morphometrics, Homology and Cladistics:
Review and Recommendations.''
}\href{http://paperpile.com/b/sTGYvp/yO2t}{\emph{Cladistics: The
International Journal of the Willi Hennig
Society}}\href{http://paperpile.com/b/sTGYvp/yO2t}{.
https://doi.org/}\href{http://dx.doi.org/10.1111/cla.12340}{10.1111/cla.12340}\href{http://paperpile.com/b/sTGYvp/yO2t}{.}

\href{http://paperpile.com/b/sTGYvp/tSIy}{Pierce, Stephanie E., Kenneth
D. Angielczyk, and Emily J. Rayfield. 2008. ``Patterns of Morphospace
Occupation and Mechanical Performance in Extant Crocodilian Skulls: A
Combined Geometric Morphometric and Finite Element Modeling Approach.''
}\href{http://paperpile.com/b/sTGYvp/tSIy}{\emph{Journal of
Morphology}}\href{http://paperpile.com/b/sTGYvp/tSIy}{ 269 (7):
840--64.}

\href{http://paperpile.com/b/sTGYvp/I0Ic}{Raup, D. M. 1961. ``THE
GEOMETRY OF COILING IN GASTROPODS.''
}\href{http://paperpile.com/b/sTGYvp/I0Ic}{\emph{Proceedings of the
National Academy of Sciences of the United States of
America}}\href{http://paperpile.com/b/sTGYvp/I0Ic}{ 47 (4): 602--9.}

\href{http://paperpile.com/b/sTGYvp/geAO}{Ruta, Marcello, Kenneth D.
Angielczyk, Jörg Fröbisch, and Michael J. Benton. 2013. ``Decoupling of
Morphological Disparity and Taxic Diversity during the Adaptive
Radiation of Anomodont Therapsids.''
}\href{http://paperpile.com/b/sTGYvp/geAO}{\emph{Proceedings. Biological
Sciences / The Royal Society}}\href{http://paperpile.com/b/sTGYvp/geAO}{
280 (1768): 20131071.}

\href{http://paperpile.com/b/sTGYvp/ZEDR}{Spriggs, Elizabeth L., Samuel
B. Schmerler, Erika J. Edwards, and Michael J. Donoghue. 2018. ``Leaf
Form Evolution in Viburnum Parallels Variation within Individual
Plants.'' }\href{http://paperpile.com/b/sTGYvp/ZEDR}{\emph{The American
Naturalist}}\href{http://paperpile.com/b/sTGYvp/ZEDR}{ 191 (2):
235--49.}

\href{http://paperpile.com/b/sTGYvp/6tNm}{Valentine, J. W. 1986.
``Fossil Record of the Origin of Baupläne and Its Implications.'' In
}\href{http://paperpile.com/b/sTGYvp/6tNm}{\emph{Patterns and Processes
in the History of Life}}\href{http://paperpile.com/b/sTGYvp/6tNm}{,
209--22.}

\href{http://paperpile.com/b/sTGYvp/Ejzr}{Wainwright, Peter C., Michael
E. Alfaro, Daniel I. Bolnick, and C. Darrin Hulsey. 2005. ``Many-to-One
Mapping of Form to Function: A General Principle in Organismal Design?''
}\href{http://paperpile.com/b/sTGYvp/Ejzr}{\emph{Integrative and
Comparative Biology}}\href{http://paperpile.com/b/sTGYvp/Ejzr}{ 45 (2):
256--62.}

\href{http://paperpile.com/b/sTGYvp/nFf7}{Wills, Matthew A. 2001.
``Morphological Disparity: A Primer.'' In
}\href{http://paperpile.com/b/sTGYvp/nFf7}{\emph{Topics in
Geobiology}}\href{http://paperpile.com/b/sTGYvp/nFf7}{, 55--144.}

\href{http://paperpile.com/b/sTGYvp/eZ3F}{Wills, Matthew A., Derek E. G.
Briggs, and Richard A. Fortey. 1994. ``Disparity as an Evolutionary
Index: A Comparison of Cambrian and Recent Arthropods.''
}\href{http://paperpile.com/b/sTGYvp/eZ3F}{\emph{Paleobiology}}\href{http://paperpile.com/b/sTGYvp/eZ3F}{
20 (02): 93--130.}

\href{http://paperpile.com/b/sTGYvp/s33b}{Wright, David F. 2017.
``Phenotypic Innovation and Adaptive Constraints in the Evolutionary
Radiation of Palaeozoic Crinoids.''
}\href{http://paperpile.com/b/sTGYvp/s33b}{\emph{Scientific
Reports}}\href{http://paperpile.com/b/sTGYvp/s33b}{ 7 (1): 13745.}

\href{http://paperpile.com/b/sTGYvp/ZQS9}{Zelditch, Miriam Leah, Donald
L. Swiderski, and H. David Sheets. 2012.
}\href{http://paperpile.com/b/sTGYvp/ZQS9}{\emph{Geometric Morphometrics
for Biologists: A Primer}}\href{http://paperpile.com/b/sTGYvp/ZQS9}{.
Academic Press.}

\href{http://paperpile.com/b/sTGYvp/ZqML}{Zou, Zhengting, and Jianzhi
Zhang. 2016. ``Morphological and Molecular Convergences in Mammalian
Phylogenetics.'' }\href{http://paperpile.com/b/sTGYvp/ZqML}{\emph{Nature
Communications}}\href{http://paperpile.com/b/sTGYvp/ZqML}{ 7
(September): 12758.}
